\documentclass[a4paper, 10pt, twocolumn]{article}
\usepackage[utf8]{inputenc}
\usepackage[german]{babel}
\selectlanguage{german}

\begin{document}

\author{themerius, no-privacy} 
\title{Soziale Marktwirtschaft 2.0 — wie die Idee einer
kapitalistischen Basisdemokratie die
Wirtschaftswelt umkrempeln könnte}
\maketitle

\begin{abstract}
Das Abstract
\end{abstract}

\section{Das Problem}
\begin{itemize}
    \item gibt schon viel ansätze, streuaktien/streubesitz
    \item im prinzip ist das system schon so, aber es gibt kein
          gleichgewicht / falsche gewichtung
\end{itemize}

\section{Mehr Mitbestimmung}
\begin{itemize}
    \item wie momentan beschaffen?
    \item was kann man ändern?
    \item mehr demokratie
    \item verteilt ist besser als zentralisiert
\end{itemize}

\section{Mögliche Konsequenzen}
\begin{itemize}
    \item u.a. vor und nachteile
\end{itemize}

\section{Realisierung}
\begin{itemize}
    \item mikroaktie
\end{itemize}

\section{Eine soziale Aktiengesellschaft}
\begin{itemize}
    \item die geschäftsform der zukunft? struktur / "kodex"?
\end{itemize}

\end{document}
